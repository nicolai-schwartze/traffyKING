\documentclass[./\jobname.tex]{subfiles}
\begin{document}

\chapter{Task Description}

This project is about simulating and optimising a circulative traffic light control. The simulation is done with  \href{https://www.dlr.de/ts/en/desktopdefault.aspx/tabid-9883/16931_read-41000/}{SUMO} which stands for Simulation of Urban MObility. SUMO provides an API that allows external programs to cast and controll a simulation. These control structures, simulation evaluation as well as the optimisation algorithms are programmed in Python 3.7. \\~\\
The project includes the following tasks: 
\begin{itemize}
	\item create infrastructure to call simulation and alter simulation parametes in SUMO
	\item construct three different traffic load scenarios
	\item three different optimisation algorithms
	\begin{itemize}
		\item NSGA2
		\item Conjugate Gradinet Descent
		\item self constructed solution
	\end{itemize}
	\item description and evaluation of the results
	\item precise technical report to ensure repeatability \\~\\
\end{itemize}

The simulation returns three parameters: 
\begin{itemize}
	\item overall waitingtime calculated as the sum of the waiting time of all cars
	\item overall number of stops calculated as the sum of the stops by all cars
	\item fairness in waitingtime calculated as the variance of the waitingtimes \\
\end{itemize}
We are looking for a solution that minimises all of these parameters. \\
All of the considered traffic load scenarios are stable. This means, that the waiting line does not grow arbitrarily larg. 

\chapter{Experiment Setup}

\section{Simulation Scenarios}

\section{Algorithms}

\subsection{Self Constructed Solution}

\subsection{NSGA-II}

\subsection{Conjugate Gradient Descent}

\end{document}
