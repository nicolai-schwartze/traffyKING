\documentclass[./\jobname.tex]{subfiles}
\begin{document}

\chapter{Task Description}

This project is about simulating and optimising a circulative traffic light control. The simulation is done with  \href{https://www.dlr.de/ts/en/desktopdefault.aspx/tabid-9883/16931_read-41000/}{SUMO} which stands for Simulation of Urban MObility. SUMO provides an API that allows external programs to cast and controll a simulation. These control structures, simulation evaluation as well as the optimisation algorithms are programmed in Python 3.7. \\~\\
The project includes the following tasks: 
\begin{itemize}
	\item create infrastructure to call simulation and alter simulation parametes in SUMO
	\item construct three different traffic load scenarios
	\begin{itemize}
		\item Manhatten Grid: 1x1, 1x3, 3x3
		\item traffic load: night, noon, rush-hour
	\end{itemize}
	\item five different optimisation algorithms
	\begin{itemize}
		\item NSGA2
		\item Conjugate Gradinet Descent
		\item Differential Evolution
		\item self created heuristic using Hill-Climbing
		\item self constructed solution
	\end{itemize}
	\item description and evaluation of the results
	\item precise technical report to ensure repeatability 
\end{itemize}

The simulation returns three parameters: 
\begin{itemize}
	\item overall waitingtime calculated as the sum of the waiting time of all cars
	\item overall number of stops calculated as the sum of the stops by all cars
	\item fairness in waitingtime calculated as the variance of the waitingtimes \\
\end{itemize}
We are looking for a solution that minimises all of these parameters. \\
All of the considered traffic load scenarios are stable. This means, that the waiting line does not grow arbitrarily larg. 

\newpage
\textbf{\underline{Problem:}} Since the cars can take a turn at every intersection, it might be possible that car gets trapped in the grid for a longer period. This aritficially increases the number of stopcounts and the waitingtime. The optimiser can not improve this situation because the traffic light has no impact on this issue. To prevent that this effect weakens the found solutions, the function evaluation must be adapted. This is done by normalising the waiting time as well as the stopcount with the length of the trip by that specific car. However this also implies, that the grid is equally spaced. 

\chapter{Simulation Infrastructur}
sumo environment, simulation runner, simulation parameter, reroute, tripinfofile and simulation result

\chapter{Algorithms}
pseudocode, implementation, characteristics, parameters, original papers reference

\chapter{Scenarios}
sumo cross, reroute probability (Abbiegewahrscheinlichekit), LP, qin max

\chapter{Experiments}

\section{Results}

\chapter{Conclustion}

\end{document}
